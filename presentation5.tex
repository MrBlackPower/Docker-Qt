%% This Beamer template is based on the one found here: https://github.com/sanhacheong/stanford-beamer-presentation, and edited to be used for Stanford ARM Lab

\documentclass[10pt]{beamer}
%\mode<presentation>{}

\usepackage{media9}
\usepackage{amssymb,amsmath,amsthm,enumerate}
\usepackage[utf8]{inputenc}
\usepackage{array}
\usepackage[parfill]{parskip}
\usepackage{graphicx}
\usepackage{caption}
\usepackage{subcaption}
\usepackage{bm}
\usepackage{amsfonts,amscd}
\usepackage[]{units}
\usepackage{listings}
\usepackage{multicol}
\usepackage{multirow}
\usepackage{tcolorbox}
\usepackage{physics}
%encoding
%--------------------------------------
\usepackage[T1]{fontenc}
\usepackage[utf8]{inputenc}
%--------------------------------------

%Portuguese-specific commands
%--------------------------------------
\usepackage[portuguese]{babel}
%--------------------------------------

%Hyphenation rules
%--------------------------------------
\usepackage{hyphenat}
\hyphenation{mate-mática recu-perar}
%--------------------------------------

% Enable colored hyperlinks
\hypersetup{colorlinks=true}

% The following three lines are for crossmarks & checkmarks
\usepackage{pifont}% http://ctan.org/pkg/pifont
\newcommand{\cmark}{\ding{51}}%
\newcommand{\xmark}{\ding{55}}%

% Numbered captions of tables, pictures, etc.
\setbeamertemplate{caption}[numbered]

%\usepackage[superscript,biblabel]{cite}
\usepackage{algorithm2e}
\renewcommand{\thealgocf}{}

% Bibliography settings
\usepackage[style=ieee]{biblatex}
\setbeamertemplate{bibliography item}{\insertbiblabel}
\addbibresource{references.bib}

% Glossary entries
\usepackage[acronym]{glossaries}
\newacronym{ML}{ML}{machine learning}
\newacronym{HRI}{HRI}{human-robot interactions}
\newacronym{RNN}{RNN}{Recurrent Neural Network}
\newacronym{LSTM}{LSTM}{Long Short-Term Memory}


\theoremstyle{remark}
\newtheorem*{remark}{Remark}
\theoremstyle{definition}

\newcommand{\empy}[1]{{\color{darkorange}\emph{#1}}}
\newcommand{\empr}[1]{{\color{cardinalred}\emph{#1}}}
\newcommand{\examplebox}[2]{
\begin{tcolorbox}[colframe=darkcardinal,colback=boxgray,title=#1]
#2
\end{tcolorbox}}

\usetheme{Stanford} 
\input{./style_files_stanford/my_beamer_defs.sty}
\logo{\includegraphics[height=0.4in]{./images/logoufjf10.png}}

% commands to relax beamer and subfig conflicts
% see here: https://tex.stackexchange.com/questions/426088/texlive-pretest-2018-beamer-and-subfig-collide
\makeatletter
\let\@@magyar@captionfix\relax
\makeatother

\newcommand{\code}[1]{\textcolor{red} {\textit{#1}}} %comentarios

\title[Reunião de Orientação 04]{Reunião de Orientação 05}
%\subtitle{Subtitle Of Presentation}

%\beamertemplatenavigationsymbolsempty

\begin{document}

\author[Modelagem Computacional]{
	\begin{tabular}{c} 
	\Large
	Igor Pires dos Santos\\
    \footnotesize \href{mailto:igor.pires@ice.ufjf.br}{igor.pires@ice.ufjf.br}\\
    \textbf{Orientador:} Rafael Bonfim
\end{tabular}
\vspace{-4ex}}

\institute{
	\vskip 5pt
	\begin{figure}
		\centering
		\begin{subfigure}[t]{0.5\textwidth}
			\centering
			\includegraphics[height=0.33in]{images/logoufjf1}
		\end{subfigure}%
		~ 
		\begin{subfigure}[t]{0.5\textwidth}
			\centering
			\includegraphics[height=0.33in]{./images/PGMC.png}
		\end{subfigure}
	\end{figure}
	\vskip 5pt
	Programa de Pós-Graduação em Modelagem Computacional\\
	Universidade Federal de Juiz de Fora\\
	\vskip 3pt
}

% \date{June 15, 2020}
\date{\today}

\begin{noheadline}
\begin{frame}\maketitle\end{frame}
\end{noheadline}

\setbeamertemplate{itemize items}[default]
\setbeamertemplate{itemize subitem}[circle]

\begin{frame}
	\frametitle{Sumário} % Table of contents slide, comment this block out to remove it
	\tableofcontents % Throughout your presentation, if you choose to use \section{} and \subsection{} commands, these will automatically be printed on this slide as an overview of your presentation
\end{frame}

\section{Afericao com Árvore Duan  Zamir}
\begin{frame}[allowframebreaks]
\frametitle{Aferição com Árvores}
	
	\begin{itemize}
		\item \textbf{Árvore proposta}
		\item Árvore extraída do artigo Duan \& Zamir.
		\item Árvore CCO N = 2,3,4,5.
		
	\end{itemize}

	\framebreak
	
	\begin{itemize}
		\item \textbf{Duan \& Zamir}
		\item 2 Artérias terminais.
		\item 6 segmentos totais (2 pares idênticos).
		\item Considerando o caso não-viscoso e  $\phi = 0$
		\item Fase um (7 variáveis) + Fase Dois (5 variáveis)
		
	\end{itemize}

\framebreak

\begin{itemize}
\item \textbf{Parâmetros de Entrada (r(cm),L(cm),$\rho$(g/$cm^3$),E(g/$cm*s^2$),f(Hz))}
\item $ f =3.65$Hz.
\item [0] = $(r = 0.65)$, $(L = 25)$,$(\rho = 0.96)$ e $(E = 4.8 * 10^6)$.
\item [1] = $(r = 0.45)$, $(L = 11)$,$(\rho = 1.134)$ e $(E = 10^7)$.
\item [2] = $(r = 0.3)$, $(L = 12)$,$(\rho = 1.172)$ e $(E = 10^7)$.
\item [3] = $(r = 0.2)$, $(L = 10)$,$(\rho = 1.235)$ e $(E = 10^7)$.

\end{itemize}

\framebreak

\begin{itemize}
	\item \textbf{Wall Thickness (h)(cm)}
	\item $ h = 0.1 * r $.
	\item [0] = 0.065.
	\item [1] = 0.045.
	\item [2] = 0.03.
	\item [3] = 0.02.
	
\end{itemize}

\framebreak

\begin{itemize}
\item \textbf{Wavespeed (c)(ang/s)}
\item $ C = \sqrt{\frac{Eh}{\rho 2 r}}$.
\item [0] = 500.
\item [1] = 664.0158940747.
\item [2] = 653.1624303415.
\item [3] = 636.2847629758.

\end{itemize}

\framebreak

\begin{itemize}
\item \textbf{Angular Frequency}
\item $ \omega = 2 \pi f$.
\item [0] = 22.9336263712.
\item [1] = 22.9336263712.
\item [2] = 22.9336263712.
\item [3] = 22.9336263712.

\end{itemize}

\framebreak

\begin{itemize}
\item \textbf{Beta <<complex>>}
\item $ \beta = \omega \frac{L}{c}$.
\item [0] = (1.1466813186  ,  0).
\item [1] = (0.3799154393  ,  0).
\item [2] = (0.4213400790  ,  0).
\item [3] = (0.3604302299  ,  0).

\end{itemize}

\framebreak

\begin{itemize}
\item \textbf{Admittance <<complex>>}
\item $ Y = \frac{\pi r^2}{\rho c}$.
\item [0] = (0.0027652560  ,  0).
\item [1] = (0.00084485573 ,  0).
\item [2] = (0.0003693547  ,  0).
\item [3] = (0.0001599158  ,  0).

\end{itemize}

\framebreak

\begin{itemize}
\item \textbf{Reflection Coefficient <<complex>>}
\item Se folha $ R = 0 $, senão $ R = \frac{Y - (Ye_r + Ye_l)}{Y + (Ye_r Ye_l)}$.
\item [0] = (0.6262367793  ,  -0.2822851808).
\item [1] = (0.3301552903  ,  -0.2838246367).
\item [2] = (0.3957123386  ,  0).
\item [3] = (0             ,  0).

\end{itemize}

\framebreak

\begin{itemize}
\item \textbf{Effective Admittance <<complex>>}
\item Se folha $ Ye = Y $, senão $ Ye = Y * \frac{(1 - R\exp{-2i\beta})}{(1 + R\exp{-2i\beta})}$.
\item [0] = (0.0066356520  ,  0.0071130215).
\item [1] = (0.0005360764  ,  0.0005730514).
\item [2] = (0.0001850690  ,  0.0001296261).
\item [3] = (0.0001599158  ,  0).

\end{itemize}
	
	
	
\end{frame}

\end{document}
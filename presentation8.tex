%% This Beamer template is based on the one found here: https://github.com/sanhacheong/stanford-beamer-presentation, and edited to be used for Stanford ARM Lab

\documentclass[10pt]{beamer}
%\mode<presentation>{}

\usepackage{media9}
\usepackage{amssymb,amsmath,amsthm,enumerate}
\usepackage[utf8]{inputenc}
\usepackage{array}
\usepackage[parfill]{parskip}
\usepackage{graphicx}
\usepackage{caption}
\usepackage{subcaption}
\usepackage{bm}
\usepackage{amsfonts,amscd}
\usepackage[]{units}
\usepackage{listings}
\usepackage{multicol}
\usepackage{multirow}
\usepackage{tcolorbox}
\usepackage{physics}
%encoding
%--------------------------------------
\usepackage[T1]{fontenc}
\usepackage[utf8]{inputenc}
%--------------------------------------

%Portuguese-specific commands
%--------------------------------------
\usepackage[portuguese]{babel}
%--------------------------------------

%Hyphenation rules
%--------------------------------------
\usepackage{hyphenat}
\hyphenation{mate-mática recu-perar}
%--------------------------------------

% Enable colored hyperlinks
\hypersetup{colorlinks=true}

% The following three lines are for crossmarks & checkmarks
\usepackage{pifont}% http://ctan.org/pkg/pifont
\newcommand{\cmark}{\ding{51}}%
\newcommand{\xmark}{\ding{55}}%

% Numbered captions of tables, pictures, etc.
\setbeamertemplate{caption}[numbered]

%\usepackage[superscript,biblabel]{cite}
\usepackage{algorithm2e}
\renewcommand{\thealgocf}{}

% Bibliography settings
\usepackage[style=ieee]{biblatex}
\setbeamertemplate{bibliography item}{\insertbiblabel}
\addbibresource{references.bib}

% Glossary entries
\usepackage[acronym]{glossaries}
\newacronym{ML}{ML}{machine learning}
\newacronym{HRI}{HRI}{human-robot interactions}
\newacronym{RNN}{RNN}{Recurrent Neural Network}
\newacronym{LSTM}{LSTM}{Long Short-Term Memory}


\theoremstyle{remark}
\newtheorem*{remark}{Remark}
\theoremstyle{definition}

\newcommand{\empy}[1]{{\color{darkorange}\emph{#1}}}
\newcommand{\empr}[1]{{\color{cardinalred}\emph{#1}}}
\newcommand{\examplebox}[2]{
	\begin{tcolorbox}[colframe=darkcardinal,colback=boxgray,title=#1]
		#2
\end{tcolorbox}}

\usetheme{Stanford} 
\input{./style_files_stanford/my_beamer_defs.sty}
\logo{\includegraphics[height=0.4in]{./images/logoufjf10.png}}

% commands to relax beamer and subfig conflicts
% see here: https://tex.stackexchange.com/questions/426088/texlive-pretest-2018-beamer-and-subfig-collide
\makeatletter
\let\@@magyar@captionfix\relax
\makeatother

\newcommand{\code}[1]{\textcolor{red} {\textit{#1}}} %comentarios

\title[Reunião de Orientação 08]{Reunião de Orientação 08}
%\subtitle{Subtitle Of Presentation}

%\beamertemplatenavigationsymbolsempty

\begin{document}
	
	\author[Modelagem Computacional]{
		\begin{tabular}{c} 
			\Large
			Igor Pires dos Santos\\
			\footnotesize \href{mailto:igor.pires@ice.ufjf.br}{igor.pires@ice.ufjf.br}\\
			\textbf{Orientador:} Rafael Bonfim \& Ruy Freitas
		\end{tabular}
		\vspace{-4ex}}
	
	\institute{
		\vskip 5pt
		\begin{figure}
			\centering
			\begin{subfigure}[t]{0.5\textwidth}
				\centering
				\includegraphics[height=0.33in]{images/logoufjf1}
			\end{subfigure}%
			~ 
			\begin{subfigure}[t]{0.5\textwidth}
				\centering
				\includegraphics[height=0.33in]{./images/PGMC.png}
			\end{subfigure}
		\end{figure}
		\vskip 5pt
		Programa de Pós-Graduação em Modelagem Computacional\\
		Universidade Federal de Juiz de Fora\\
		\vskip 3pt
	}
	
	% \date{June 15, 2020}
	\date{\today}
	
	\begin{noheadline}
		\begin{frame}\maketitle\end{frame}
	\end{noheadline}
	
	\setbeamertemplate{itemize items}[default]
	\setbeamertemplate{itemize subitem}[circle]
	
	\begin{frame}
		\frametitle{Sumário} % Table of contents slide, comment this block out to remove it
		\tableofcontents % Throughout your presentation, if you choose to use \section{} and \subsection{} commands, these will automatically be printed on this slide as an overview of your presentation
	\end{frame}
	
	\section{Introdução}
	\begin{frame}[allowframebreaks]
		\frametitle{Introdução}
		
		\begin{itemize}
			\item A construção de modelos de árvores arteriais é importante para a realização de estudos hemodinâmicos. Neste trabalho, apresentam-se: 
			\item (i) um esquema analítico para o cálculo das características locais das ondas de fluxo e pressão em modelos de árvores arteriais 1D 
			\item (ii) um ambiente computacional desenvolvido para a simulação e visualização dos resultados no tocante à construção de modelos e estudos hemodinâmicos. Os resultados obtidos neste trabalho estão condizentes com dados numéricos relatados na literatura.
			
		\end{itemize}
		
		\framebreak
		
		\begin{itemize}
			\item A propagação de ondas em um tubo é governada pela equação de onda para a pressão $p(x,t)$ e volume de fluxo $q(x,t)$, ambas sendo funções no tempo $t$ e coordenada axial $x$ ao longo do tubo.
		\end{itemize}
		

		\begin{equation}
		\frac{\partial q}{\partial t} = -cY \frac{\partial p}{\partial x}  
		\label{01_p}
		\end{equation}
		
		\begin{equation}
		\frac{\partial p}{\partial t} = -\frac{c}{Y} \frac{\partial p}{\partial x}  
		\label{02_q}
		\end{equation}
		
		\framebreak
		\begin{itemize}
			\item Para uma onda harmônica simples Eq.\ref{01_p} e Eq.\ref{02_q} ficam na forma:
		\end{itemize}
		

		\begin{equation}
		p = \bar{p}_0 \exp{i\omega(t - x/c)} + R  \bar{p}_0 \exp{i\omega(t - 2L/c + x/c)}
		\label{03_p}
		\end{equation}
		
		\begin{equation}
		q = Y(\bar{p}_0 \exp{i\omega(t - x/c)} -  R  \bar{p}_0 \exp{i\omega(t - 2L/c + x/c)})
		\label{04_1}
		\end{equation}
		
		\framebreak
		
		\begin{itemize}
			\item Para definir os valores do coeficiente de reflexão $R$ e pressão média $\bar{p}_0$ de cada segmento foi utilizado o modelo matemático descrito por Duan \& Zamir (1995).
			\item Este modelo matemático possui uma complexidade pequena e é capaz de ser iterado muito rapidamente para modelos geométricos numerosos.
		\end{itemize}
		
	\end{frame}
	
	\section{Modelo Matemático}
	\begin{frame}[allowframebreaks]
		\frametitle{Modelo Matemático}
		
		\begin{itemize}
			\item Inicialmente se definem as propriedades características de cada segmento :
			
			\item Espessura da parede ($h$):		
			\item 
					\begin{equation}
					h = 0.1 \times r
					\label{05}
					\end{equation}
			\item Velocidade de Onda ($c$):
			\item 
					\begin{equation}
					 c = \sqrt{\frac{Eh}{\rho 2 r}}
					\label{06}
					\end{equation}
		\end{itemize}
		
		\framebreak
		
		\begin{itemize}
			\item Velocidade angular ($\omega$):
			\item 
					\begin{equation}
					\omega = 2 \pi f
					\label{07}
					\end{equation}
			\item Beta ($\beta$):
			\item 
					\begin{equation}
					 \beta = \omega \frac{L}{c}
					\label{08}
					\end{equation}
			\item Admitância Característica ($Y$):
			\item  
					\begin{equation}
					 Y = \frac{\pi r^2}{\rho c}
					\label{09}
					\end{equation}
			
		\end{itemize}
		
		
		\framebreak
		
		\begin{itemize}
			\item \textbf{Caso 2:} Ângulo de fase
			\item Módulo de Young ($E_c$):
			\item 
					\begin{equation}
					  E_c = \|E_c\|\exp{i\phi}
					\label{12}
					\end{equation}
			\item
			\item Ângulo de Fase ($\phi$):
			\item 
					\begin{equation}
					 \phi = \phi_0 (1 - \exp{-w})
					\label{13}
					\end{equation}
			
		\end{itemize}
		
		
		\framebreak
		
		\begin{itemize}
			\item \textbf{Caso 3:} Viscoso
			\item Velocidade de Onda Viscosa ($c_v$):
			\item 
					\begin{equation}
					  c_v = c\sqrt{\epsilon}
					\label{10}
					\end{equation}
			\item
			\item Admitância ($Y_v$):
			\item 
					\begin{equation}
					Y_v = Y\sqrt{\epsilon}
					\label{11}
					\end{equation}
			
		\end{itemize}
		
		\framebreak
		
		\begin{itemize}
			\item Alpha ($\alpha$):
			\item 
					\begin{equation}
					  \alpha = R \sqrt{\frac{\omega \rho}{\mu}}
					\label{11}
					\end{equation}
			\item Fator Viscoso ($\epsilon$):
			\item 
					\begin{equation}
					 \epsilon = 1 - F_{10}(\alpha)
					\label{11}
					\end{equation}
			\item Função auxiliar do Fator Viscoso ($ F_{10}$):
			\item 
					\begin{equation}
				    F_{10}(\alpha) = 1 - 2\alpha \sqrt{i}
					\label{11}
					\end{equation}
			
		\end{itemize}
		
		\framebreak
		
		\begin{itemize}
			\item \textbf{Reflection Coefficient <<complex>> e Admittance <<complex>>}
			\item Se folha $ R = 0 $, senão $ R = \frac{Y - (Ye_r + Ye_l)}{Y + (Ye_r Ye_l)}$.
			\item Se folha $ Ye = Y $, senão $ Ye = Y * \frac{(1 - R\exp{-2i\beta})}{(1 + R\exp{-2i\beta})}$.
			\item \textbf{Medium Pressure <<complex>>}
			\item Se raiz $ \bar{P} = \bar{P}_0 $, senão $ \bar{P} = \bar{P}_f * \frac{((1 + R_f)\exp{-i\beta_f})}{(1 + R\exp{-2i\beta})}$.
		\end{itemize}
		
		\framebreak
		
		\begin{itemize}
			\item \textbf{Pressure <<complex>>}
			\item $ P = \bar{P} * ( \exp{-i\beta X} + R\exp{-i2\beta}\exp{i\beta X})$.
			
			
		\end{itemize}
		
		\begin{itemize}
			\item \textbf{Flow <<complex>>}
			\item $ Q = M \bar{P} * (\exp{-i\beta X} - R\exp{-i2\beta}\exp{i\beta X})$.
			\item $ M = \frac{Y}{Y_r}$
			
		\end{itemize}
			
\end{frame}

	\section{Exemplo}
	\begin{frame}[allowframebreaks]
		\frametitle{Exemplo}
		
		\begin{itemize}
			\item Considera-se o exemplo de árvore arterial extraído extaído de Duan \& Zamir (1995).
			\item \begin{center}
						 \includegraphics[width=0.3\textwidth]{images/IGU2.png}
					 \end{center}			 
			
			
		\end{itemize}
		
		\framebreak

			\textbf{Duan \& Zamir} \\
			
			A árvore arterial conta com:
		
		\begin{itemize}
			\item 2 Artérias terminais.
			\item 6 segmentos totais (2 pares idênticos).
			\item Considerando o caso não-viscoso com ângulo de fase $\phi = 0$
			\item Observa-se o acoplamento feito para o caso viscoso e a variação do ângulo de fase.
			
		\end{itemize}
		
		O exemplo considera ainda o caso viscoso, o não viscoso e  $\phi \in {0,4,8,12}$. O algoritmo proposto se dividiu em duas partes, a primeira com 7 variáveis a se definir e a segunda com 5. Para demonstrar o modelo matemático o caso não viscoso com $\phi = 0$ foi escolhido.
		
		\framebreak
		
		\begin{itemize}
			\item \textbf{Parâmetros de Entrada (r(cm),L(cm),$\rho$(g/$cm^3$),E(g/$cm*s^2$),f(Hz),$\epsilon$,$\mu_0$,$\phi_0$)}
			\item $ f =3.65$Hz,$\epsilon = 0$,$\mu_0 = 0$,$\phi_0 = 0$.
			\item [0] = $(r = 0.65)$, $(L = 25)$,$(\rho = 0.96)$ e $(E = 4.8 * 10^6)$.
			\item [1] = $(r = 0.45)$, $(L = 11)$,$(\rho = 1.134)$ e $(E = 10^7)$.
			\item [2] = $(r = 0.3)$, $(L = 12)$,$(\rho = 1.172)$ e $(E = 10^7)$.
			\item [3] = $(r = 0.2)$, $(L = 10)$,$(\rho = 1.235)$ e $(E = 10^7)$.
			
		\end{itemize}
		
		\framebreak

		\begin{itemize}
			\item \textbf{Wall Thickness (h)(cm)}
			\item $ h = 0.1 * r $.
			\item [0] = 0.065.
			\item [1] = 0.045.
			\item [2] = 0.03.
			\item [3] = 0.02.
			
		\end{itemize}
		
		\framebreak
		
		\begin{itemize}
			\item \textbf{Wavespeed (cm/s)}
			\item $ C = \sqrt{\frac{Eh}{\rho 2 r}}$.
			\item [0] = 500.
			\item [1] = 664.0158940747.
			\item [2] = 653.1624303415.
			\item [3] = 636.2847629758.
		
		\end{itemize}
		
		\framebreak
		
		\begin{itemize}
		\item \textbf{Angular Frequency (ang/s)}
		\item $ \omega = 2 \pi f$.
		\item [0] = 22.9336263712.
		\item [1] = 22.9336263712.
		\item [2] = 22.9336263712.
		\item [3] = 22.9336263712.
		
		\end{itemize}
		
		\framebreak
		
		\begin{itemize}
		\item \textbf{Beta <<complex>>}
		\item $ \beta = \omega \frac{L}{c}$.
		\item [0] = (1.1466813186  ,  0).
		\item [1] = (0.3799154393  ,  0).
		\item [2] = (0.4213400790  ,  0).
		\item [3] = (0.3604302299  ,  0).
		
		\end{itemize}
		
		\framebreak
		
		\begin{itemize}
		\item \textbf{Admittance <<complex>>}
		\item $ Y = \frac{\pi r^2}{\rho c}$.
		\item [0] = (0.0027652560  ,  0).
		\item [1] = (0.00084485573 ,  0).
		\item [2] = (0.0003693547  ,  0).
		\item [3] = (0.0001599158  ,  0).
		
		\end{itemize}
		
		\framebreak
		
		\begin{itemize}
		\item \textbf{Reflection Coefficient <<complex>>}
		\item Se folha $ R = 0 $, senão $ R = \frac{Y - (Ye_r + Ye_l)}{Y + (Ye_r Ye_l)}$.
		\item [0] = (0.6262367793  ,  -0.2822851808).
		\item [1] = (0.3301552903  ,  -0.2838246367).
		\item [2] = (0.3957123386  ,  0).
		\item [3] = (0             ,  0).
		
		\end{itemize}
		
		\framebreak

	\begin{itemize}
		\item \textbf{Effective Admittance <<complex>>}
		\item Se folha $ Ye = Y $, senão $ Ye = Y * \frac{(1 - R\exp{-2i\beta})}{(1 + R\exp{-2i\beta})}$.
		\item [0] = (0.0066356520  ,  0.0071130215).
		\item [1] = (0.0005360764  ,  0.0005730514).
		\item [2] = (0.0001850690  ,  0.0001296261).
		\item [3] = (0.0001599158  ,  0).
	
	\end{itemize}
		
		\framebreak

	\begin{itemize}
		\item \textbf{Medium Pressure <<complex>>}
		\item Se raiz $ \bar{P} = \bar{P}_0 $, senão $ \bar{P} = \bar{P}_f * \frac{((1 + R_f)\exp{-i\beta_f})}{(1 + R\exp{-2i\beta})}$.
		\item [0] = (1.0  ,  0.0).
		\item [1] = (0.0005360764  ,  0.0005730514).
		\item [2] = (0.0001850690  ,  0.0001296261).
		\item [3] = (0.0001599158  ,  0).
	
	\end{itemize}
		
\end{frame}
	
	
	
	\section{Modelo Computacional}
	\begin{frame}[allowframebreaks]
		\frametitle{Modelo Computacional}
		
		\begin{itemize}
			
			\item \includegraphics[width=0.5\textwidth]{images/WiseElement.png}
			
			\item Um elemento inteligente (\textit{WiseElement}) possui duas estruturas básicas. A \textit{WiseStructure} que representa os dados dispostos no padrão \textit{VTK}(\textit{Visualization Toolkit}) e \textit{DataStructure} que representa os dados abstratos disponíveis na estrutura.
			
		\end{itemize}
		
		\framebreak	
		
		\begin{itemize}
			
			\item \includegraphics[width=0.6\textwidth]{images/WiseElements.png}
			
			\item Todos os elementos inteligentes recebem a estrutura básica \textit{WiseStructure} por herança e requer por meio de funções virtuais a definição de métodos que possibilitem a manipulação dos dados abstratos.
			
		\end{itemize}
		
		\framebreak
		
		\begin{itemize}
			\item Os elementos inteligentes são regidos por uma máquina de estados, aonde os estados representam condições esperadas do elemento inteligente e as transições indicam ações tomadas com o elemento inteligente.
		\end{itemize}
		
		\framebreak
		
		\begin{center}
			\includegraphics[width=0.6\textwidth]{images/WiseElementStatus.png}
		\end{center}
		
		\framebreak
		
		\begin{center}
			\includegraphics[width=0.5\textwidth]{images/WiseElementHot.png}
		\end{center}
		
		\begin{itemize}
			\item 	Neste estado espera-se que um elemento possua consigo ambas as estruturas presentes no elemento inteligente, seus dados abstratos e a estrutura inteligente \textit{WiseStructure}.
		\end{itemize}
		
		\framebreak
		\begin{center}
			\item \includegraphics[width=0.5\textwidth]{images/WiseElementWarming.png}
		\end{center}
		
		\begin{itemize}
			\item O estado \textit{Warming} é equivalente ao estado \textit{Cooling} e \textit{Raw}.
			\item Estes estados indicam que somente a estrutura inteligente deste elemento está presente. No caso de um elemento no estado \textit{Raw}, não é esperado que a estrutura completa esteja presente nesta estrutura.
			\item Os outros estados indicam que o elemento está completamente carregado na estrutura inteligente e aguarda esfriamento ou aquecimento, processo de armazenar e recuperar arquivos em HD.
		\end{itemize}
		
		\framebreak
		
		\begin{center}
			\includegraphics[width=0.5\textwidth]{images/WiseElementCold.png}
		\end{center}
		
		\begin{itemize}
			\item Espera-se que os elementos neste estado estejam salvos em HD.
		\end{itemize}
		
		\framebreak
		
		\begin{itemize}
			\item Utilizando o padrão de fábricas para criar e manipular elementos inteligentes, o seguinte fluxo de trabalho foi idealizado:
		\end{itemize}
		\begin{flushright}
			
			\item \includegraphics[width=0.5\textwidth]{images/WiseElementWorkflow.png}
			
		\end{flushright}
		
		\framebreak
	
		\begin{itemize}
				\item A primeira fábrica \textit{WiseElementFactory} é responsável por criar corretamente cada tipo de elemento inteligente.
				\item A fábrica \textit{WiseIterationFactory} tem a função de utilizar os dados abstratos de um elemento inteligente com a finalidade de executar algum algoritmo.
				\item Finalmente, a fábrica \textit{GraphicFactory} gera os objetos capazes de se desenhar com diretivas OpenGL \textit{GraphicObject}.
		\end{itemize}	
		
		\framebreak	
		
		\framebreak
		
		\begin{itemize}
				\item O objeto inteligente \textit{WiseObject} é o objeto capaz de executar todo esse fluxo de trabalho.
		\end{itemize}	
		
		\begin{center}
			
			 \includegraphics[width=0.7\textwidth]{images/WiseObject.png}
			
		\end{center}
		
		\framebreak
	
			\begin{itemize}
				\item 	Cada objeto inteligente contém uma coleção de elementos inteligentes
			\end{itemize}		
		
		\begin{center}
			
			\item \includegraphics[width=0.6\textwidth]{images/WiseCollection.png}
			
		\end{center}
		\framebreak
		
		\begin{itemize}
			\item Pode-se criar um objeto inteligente com um elemento inteligente, portanto é possível construir um objeto inteligente à partir de qualquer instância de elemento inteligente e sua fábrica.
			\item A estrutura \textit{Forno} é um ponteiro para um elemento que estará sempre no estado \textit{Hot}, ele representa o último elemento gerado pela iteração e é utilizado a cada nova iteração.
			\item Finalmente, a estrutura \textit{Freezer} é responsável por armazenar os elementos em cache e manter o seu registro.
		\end{itemize}
		
		\framebreak
		
		\begin{itemize}
				\item É possível também que seja incluída a estrutura de visualização do objeto inteligente:
		\end{itemize}		
		
		\begin{center}
			
			\item \includegraphics[width=0.6\textwidth]{images/GraphicModel.png}
			
		\end{center}
		\framebreak
		
		\begin{itemize}
			\item Os objetos gráficos são objetos que possuem uma lista de elementos gráficos com valores e é capas de desenhá-los em um quadro OpenGL.
			\item O modelo gráfico \textit{GraphicModel} irá manter a coleção de objetos gráficos em memória conforme a necessidade para visualização, mantendo uma quantidade mínima de objetos em memória a todo momento.
		\end{itemize}		
		
		\framebreak
		\begin{center}
			
			\item \includegraphics[width=0.6\textwidth]{images/GraphicObjects.png}
			
		\end{center}
		
		\framebreak
		
		\begin{itemize}
				\item Objetos gráficos também possuem uma máquina de estados que dita se a estrutura está presente em memória ou armazenada em cache:
		\end{itemize}		
		
		\begin{center}
			
			\item \includegraphics[width=0.6\textwidth]{images/GraphicObjectStatus.png}
			
		\end{center}
		
		\framebreak
		\begin{itemize}
				\item Os elementos gráficos são todas as instâncias que implementam a classe abstrata \textit{GraphicElement}:
		\end{itemize}		
		
		\begin{center}
			
			\item \includegraphics[width=0.6\textwidth]{images/GraphicElements.png}
			
		\end{center}
		
		\begin{itemize}
			\item Os elementos gráficos armazenam os pontos necessários para desenhá-lo e um valor associado. Este valor associado corresponde à algum valor armazenado em um ponto, uma linha, uma célula ou um campo da \textit{WiseStructure}.
		\end{itemize}
		
		\framebreak
	
		\begin{itemize}
				\item Os objetos inteligentes \textit{WiseObject} possuem também uma máquina de estados:
		\end{itemize}		
		
		\begin{center}
			
			\item \includegraphics[width=0.6\textwidth]{images/WiseObjectStatus.png}
			
		\end{center}
		
		\framebreak
		
		\begin{itemize}
			\item Todas as transições de estados e as ações que podem ser executadas sobre um objeto inteligente são interações do usuário. Ao se criar o objeto somente o elemento inicial e sua cópia estarão presentes no \textit{Forno} e no \textit{Freezer}, respectivamente.
			\item No estado inicial \textit{Ready} o objeto é capaz de incluir suas fábricas de iteração e gráficas.
			\item Após a inclusão de uma fábrica de iteração o objeto pode avançar para o estado \textit{Set}. Neste estado os parâmetros de iteração são adicionados à estrutura \textit{WiseStructure} e podem ser editados pelo usuário, parâmetros como frequência, viscosidade e ângulo de fase.
			\item Pode-se arbitrariamente definir o fim das iterações e enviar o objeto para o estado \textit{Finished}, que impossibilitará o objeto de continuar iterando.
			\item Finalmente, todos os estados podem levar ao estado \textit{Crashed} que indica o mau funcionamento do objeto.
		\end{itemize}
		
		\framebreak
		
			\begin{itemize}
					\item Com todas as estruturas básicas definidas, o pacote de classes que organiza or objetos e elementos inteligentes e suas necessidades intitula-se \textit{WiseThreadPool}.
			\end{itemize}		
		
		\begin{center}
			
			\item \includegraphics[width=0.6\textwidth]{images/WiseThreadPool.png}
			
		\end{center}
		
		\framebreak
		
		\begin{itemize}
			\item Os objetos e elementos inteligentes organizam-se em projetos inteligentes, \textit{WiseProject}.
			\item A classe \textit{WiseThreadPool} é responsável por alocar estes projetos e receber suas demandas, resfriar ou aquecer objetos. É responsável também pelas demandas feitas pelo usuário através de linhas de comando.
		\end{itemize}
		
		\framebreak
		
	\begin{itemize}
			\item Cada uma das classes à seguir está contida em uma \textit{thread} própria e tem o próprio \textit{loop} de iteração.
		\end{itemize}		
		
		\begin{center}
			
			\item \includegraphics[width=0.6\textwidth]{images/WiseThreadPool2.png}
			
		\end{center}
		
		\framebreak

	\begin{itemize}
			\item Ao receber uma linha de comando o objeto \textit{WiseThreadPool} enviar a mensagem e o projeto atual à uma instância de \textit{WiseConsole}, que é responsável por interpretar o comando.
			\item Caso seja um processo que utilize o disco rígido, uma instância \textit{WiseIO} será necessária para executar o comando. 
			\item Caso seja um processo de iteração, uma instância \textit{WiseProcessor} será utilizada.
			\item Para os restantes dos casos a própria classe \textit{WiseConsole} será utilizada para finalizar o comando. A alteração de parâmetros ou sua escala cai neste último caso.
		\end{itemize}		
		
		\begin{center}
			
			\item \includegraphics[width=0.6\textwidth]{images/WiseThreaPoolCMD.png}
			
		\end{center}
		
		\framebreak
		\begin{center}
			
			\item \includegraphics[width=0.6\textwidth]{images/WiseThreaPoolCMDSub.png}
			
		\end{center}
		
		\framebreak
		
		\begin{itemize}
				\item O resfriamento ou aquecimento de elementos envolve sempre uma instância de \textit{WiseIO}.
		\end{itemize}		
		
		\begin{center}
			
			\item \includegraphics[width=0.6\textwidth]{images/WiseThreadPoolHeating.png}
			
		\end{center}
		
		\framebreak
		
		\begin{itemize}
			\item Um \textit{WiseThreadPool} irá conter todas as instâncias de \textit{WiseIO}, \textit{WiseConsole} e \textit{WiseProcessor} e está preparado para utilizar mais de uma instância destes objetos por vez.
			\item Essencialmente isto significa que os objetos e elementos armazenados pelo projeto inteligente, \textit{WiseProject}, podem ser acessados pelos diferentes processadores independentemente.
			
			\item Isto dá ao usuário uma imensa liberdade na hora de decidir executar algum algoritmo concorrentemente, entretanto requer cuidados especiais na hora de utilizar os objetos.
			
			\item Como os objetos podem ser acessados por mais de um processador, ele é bloqueado por uso.
		\end{itemize}
		
	\end{frame}
	
	\section{Resultados}
	\begin{frame}[allowframebreaks]
		\frametitle{Resultados}
		\begin{center}
			\includegraphics[width=0.4\textwidth]{images/48.png}
			\includegraphics[width=0.4\textwidth]{images/49.png}
			\includegraphics[width=0.4\textwidth]{images/50.png}
			\includegraphics[width=0.4\textwidth]{images/51.png}
		\end{center}
		
		\framebreak
		
		\begin{center}
			\includegraphics[width=0.4\textwidth]{images/52.png}
			\includegraphics[width=0.4\textwidth]{images/53.png}
			\includegraphics[width=0.4\textwidth]{images/54.png}
			\includegraphics[width=0.4\textwidth]{images/55.png}
		\end{center}
	
		\framebreak
			
		\begin{center}
			\includegraphics[width=0.4\textwidth]{images/56.png}
			\includegraphics[width=0.4\textwidth]{images/57.png}
			\includegraphics[width=0.4\textwidth]{images/58.png}
			\includegraphics[width=0.4\textwidth]{images/59.png}
		\end{center}
		
		\framebreak
		
		\begin{center}
			\includegraphics[width=0.4\textwidth]{images/60.png}
			\includegraphics[width=0.4\textwidth]{images/61.png}
			\includegraphics[width=0.4\textwidth]{images/62.png}
			\includegraphics[width=0.4\textwidth]{images/63.png}
		\end{center}
	
	\end{frame}
	
	\section{Dissertação}
	\begin{frame}[allowframebreaks]
		\frametitle{Dissertação}
		
		\begin{itemize}
			\item Ferramenta Computacional
			\item Estrutura de dados
			\item \textbf{dividir em elemento inteligente, objeto inteligente e objeto gráfico?}
			\item  (Sinais e slots, paralelização)
			\item Lista de comandos
			\item Interface gráfica
			\item Resultados
			\item Fluxo
			\item Pressão
			\item Conclusão
		\end{itemize}
		
	\end{frame}
	
	\section{Cronograma}
	\begin{frame}[allowframebreaks]
		\frametitle{Cronograma}
		
		\begin{itemize}
			\item Mar: Ajustes dissertação + Fim da dissertação
			\item Abr: Ajustes finais
			\item Mai: Ajustes finais
		\end{itemize}
		
	\end{frame}
	
\end{document}
